% This is free and unencumbered software released into the public domain.

% Anyone is free to copy, modify, publish, use, compile, sell, or
% distribute this software, either in source code form or as a compiled
% binary, for any purpose, commercial or non-commercial, and by any
% means.

% In jurisdictions that recognize copyright laws, the author or authors
% of this software dedicate any and all copyright interest in the
% software to the public domain. We make this dedication for the benefit
% of the public at large and to the detriment of our heirs and
% successors. We intend this dedication to be an overt act of
% relinquishment in perpetuity of all present and future rights to this
% software under copyright law.

% THE SOFTWARE IS PROVIDED "AS IS", WITHOUT WARRANTY OF ANY KIND,
% EXPRESS OR IMPLIED, INCLUDING BUT NOT LIMITED TO THE WARRANTIES OF
% MERCHANTABILITY, FITNESS FOR A PARTICULAR PURPOSE AND NONINFRINGEMENT.
% IN NO EVENT SHALL THE AUTHORS BE LIABLE FOR ANY CLAIM, DAMAGES OR
% OTHER LIABILITY, WHETHER IN AN ACTION OF CONTRACT, TORT OR OTHERWISE,
% ARISING FROM, OUT OF OR IN CONNECTION WITH THE SOFTWARE OR THE USE OR
% OTHER DEALINGS IN THE SOFTWARE.

% For more information, please refer to <https://unlicense.org>



% First, define the document here, this is appropriate: 
% \documentclass[12pt]{article}

% This is the "pre-include" section. Define your first and last name, title, and the appropriate instructor's name and class
% \newcommand{\lastname}{YourLastName}
% \newcommand{\firstname}{YourFirstName}
% \newcommand{\instructorfull}{InstructorName}
% \newcommand{\className}{className}
% \newcommand{\titleName}{Title}
% 
% Now, include this file with the command:
% % This is free and unencumbered software released into the public domain.

% Anyone is free to copy, modify, publish, use, compile, sell, or
% distribute this software, either in source code form or as a compiled
% binary, for any purpose, commercial or non-commercial, and by any
% means.

% In jurisdictions that recognize copyright laws, the author or authors
% of this software dedicate any and all copyright interest in the
% software to the public domain. We make this dedication for the benefit
% of the public at large and to the detriment of our heirs and
% successors. We intend this dedication to be an overt act of
% relinquishment in perpetuity of all present and future rights to this
% software under copyright law.

% THE SOFTWARE IS PROVIDED "AS IS", WITHOUT WARRANTY OF ANY KIND,
% EXPRESS OR IMPLIED, INCLUDING BUT NOT LIMITED TO THE WARRANTIES OF
% MERCHANTABILITY, FITNESS FOR A PARTICULAR PURPOSE AND NONINFRINGEMENT.
% IN NO EVENT SHALL THE AUTHORS BE LIABLE FOR ANY CLAIM, DAMAGES OR
% OTHER LIABILITY, WHETHER IN AN ACTION OF CONTRACT, TORT OR OTHERWISE,
% ARISING FROM, OUT OF OR IN CONNECTION WITH THE SOFTWARE OR THE USE OR
% OTHER DEALINGS IN THE SOFTWARE.

% For more information, please refer to <https://unlicense.org>



% First, define the document here, this is appropriate: 
% \documentclass[12pt]{article}

% This is the "pre-include" section. Define your first and last name, title, and the appropriate instructor's name and class
% \newcommand{\lastname}{YourLastName}
% \newcommand{\firstname}{YourFirstName}
% \newcommand{\instructorfull}{InstructorName}
% \newcommand{\className}{className}
% \newcommand{\titleName}{Title}
% 
% Now, include this file with the command:
% % This is free and unencumbered software released into the public domain.

% Anyone is free to copy, modify, publish, use, compile, sell, or
% distribute this software, either in source code form or as a compiled
% binary, for any purpose, commercial or non-commercial, and by any
% means.

% In jurisdictions that recognize copyright laws, the author or authors
% of this software dedicate any and all copyright interest in the
% software to the public domain. We make this dedication for the benefit
% of the public at large and to the detriment of our heirs and
% successors. We intend this dedication to be an overt act of
% relinquishment in perpetuity of all present and future rights to this
% software under copyright law.

% THE SOFTWARE IS PROVIDED "AS IS", WITHOUT WARRANTY OF ANY KIND,
% EXPRESS OR IMPLIED, INCLUDING BUT NOT LIMITED TO THE WARRANTIES OF
% MERCHANTABILITY, FITNESS FOR A PARTICULAR PURPOSE AND NONINFRINGEMENT.
% IN NO EVENT SHALL THE AUTHORS BE LIABLE FOR ANY CLAIM, DAMAGES OR
% OTHER LIABILITY, WHETHER IN AN ACTION OF CONTRACT, TORT OR OTHERWISE,
% ARISING FROM, OUT OF OR IN CONNECTION WITH THE SOFTWARE OR THE USE OR
% OTHER DEALINGS IN THE SOFTWARE.

% For more information, please refer to <https://unlicense.org>



% First, define the document here, this is appropriate: 
% \documentclass[12pt]{article}

% This is the "pre-include" section. Define your first and last name, title, and the appropriate instructor's name and class
% \newcommand{\lastname}{YourLastName}
% \newcommand{\firstname}{YourFirstName}
% \newcommand{\instructorfull}{InstructorName}
% \newcommand{\className}{className}
% \newcommand{\titleName}{Title}
% 
% Now, include this file with the command:
% % This is free and unencumbered software released into the public domain.

% Anyone is free to copy, modify, publish, use, compile, sell, or
% distribute this software, either in source code form or as a compiled
% binary, for any purpose, commercial or non-commercial, and by any
% means.

% In jurisdictions that recognize copyright laws, the author or authors
% of this software dedicate any and all copyright interest in the
% software to the public domain. We make this dedication for the benefit
% of the public at large and to the detriment of our heirs and
% successors. We intend this dedication to be an overt act of
% relinquishment in perpetuity of all present and future rights to this
% software under copyright law.

% THE SOFTWARE IS PROVIDED "AS IS", WITHOUT WARRANTY OF ANY KIND,
% EXPRESS OR IMPLIED, INCLUDING BUT NOT LIMITED TO THE WARRANTIES OF
% MERCHANTABILITY, FITNESS FOR A PARTICULAR PURPOSE AND NONINFRINGEMENT.
% IN NO EVENT SHALL THE AUTHORS BE LIABLE FOR ANY CLAIM, DAMAGES OR
% OTHER LIABILITY, WHETHER IN AN ACTION OF CONTRACT, TORT OR OTHERWISE,
% ARISING FROM, OUT OF OR IN CONNECTION WITH THE SOFTWARE OR THE USE OR
% OTHER DEALINGS IN THE SOFTWARE.

% For more information, please refer to <https://unlicense.org>



% First, define the document here, this is appropriate: 
% \documentclass[12pt]{article}

% This is the "pre-include" section. Define your first and last name, title, and the appropriate instructor's name and class
% \newcommand{\lastname}{YourLastName}
% \newcommand{\firstname}{YourFirstName}
% \newcommand{\instructorfull}{InstructorName}
% \newcommand{\className}{className}
% \newcommand{\titleName}{Title}
% 
% Now, include this file with the command:
% \include{mla.tex}
% 
\usepackage[american]{babel} 
\usepackage{csquotes} 
\usepackage[hidelinks]{hyperref} % Links otherwise get in the way, so we hide them. Do want you want...
\usepackage[style=mla-new, backend=biber]{biblatex} % Your weird for not using this stuff...(if not why are you really using latex)
\usepackage{datetime}
\usepackage[letterpaper, margin=1.0in]{geometry} %Per mla suggestion, but some instructors deviate...
\usepackage{fancyhdr}
\usepackage{setspace}
\usepackage{enotez} % The modern version of the "endnotes" package. 
\setenotez{backref, counter-format=alph, list-heading = \begin{center}#1\end{center}} % Here we enable backlinks and switch the counter to alphabetical since usually there are less end-notes and to differentiate between them and footnotes. Also a neat trick, just change the list-heading field to be unformated and centered per MLA standard.

% We like Zapf's Palatino font for both body text and math, otherwise use something else as requested by instructor
\usepackage{newpxtext,newpxmath}

\DeclareLanguageMapping{american}{american-mla}
\newdateformat{dateMLA}{\THEDAY\ \monthname[\THEMONTH] \THEYEAR} % MLA has a specific date standard

\setstretch{2.25} % Experimentally derived spacing for word doc style. Not perfect, but at least lines per page are consistant
\setlength{\parskip}{8pt} % Typically, Word defaults to 8pt line skips, which here is the 'de facto' (much like line-spacing)

\fancyhf{}
\rhead{\lastname~\thepage}

\renewcommand{\headrulewidth}{0pt} % No headrules for the headers and footers, unless you (and your instructor) are into that...
\renewcommand{\footrulewidth}{0pt}

% Ensure header 0.5in from top edge: 12pt font is 1/6 inch
\setlength\headsep{0.333in}
% Set the header height (complains otherwise)
\setlength\headheight{14.49998pt}

\pagestyle{fancy} % Every page uses the same style.

% Custom format of bibliography for MLA
\newcommand{\printformatedbibliography}{\newpage\begin{center}Works Cited\end{center}\printbibliography[heading=none]}

% Per MLA requirements, endnotes should appear on a extra page before the bibliography with a header.
\newcommand{\printnotes}{\newpage\printendnotes}

% Useful if you want to introduce an author by their full name initially.
\newrobustcmd*{\citefullname}{\AtNextCite{\DeclareNameAlias{labelname}{given-family}}\citeauthor} 

\newenvironment{paper}{
\firstname~\lastname \\ 
\instructorfull\\ 
\className
% \dateMLA{\today}
% \begin{center}\titleName\end{center}
\setlength\parindent{0.5in}
}{
%\printnotes % If your not using endnotes, just comment this out...
\printformatedbibliography % Most likely, your should keep this
}

\setlength\parindent{0.5in}
\raggedright


% The "post-include" section. Simply include your bibliography database file and begin the document. Below is the example code
%
% \bibliography{sources.bib} %<-- BTY, you have to do this post-include.
% \begin{document}
% \begin{paper}
% Here be a footnote \footnote{EEEEE! Sup}. Here be a endnote\endnote{EEEEEEEEEEEEEEEEEE! We dont use those often}
% \end{paper}
% \end{document}
% 
% 
\usepackage[american]{babel} 
\usepackage{csquotes} 
\usepackage[hidelinks]{hyperref} % Links otherwise get in the way, so we hide them. Do want you want...
\usepackage[style=mla-new, backend=biber]{biblatex} % Your weird for not using this stuff...(if not why are you really using latex)
\usepackage{datetime}
\usepackage[letterpaper, margin=1.0in]{geometry} %Per mla suggestion, but some instructors deviate...
\usepackage{fancyhdr}
\usepackage{setspace}
\usepackage{enotez} % The modern version of the "endnotes" package. 
\setenotez{backref, counter-format=alph, list-heading = \begin{center}#1\end{center}} % Here we enable backlinks and switch the counter to alphabetical since usually there are less end-notes and to differentiate between them and footnotes. Also a neat trick, just change the list-heading field to be unformated and centered per MLA standard.

% We like Zapf's Palatino font for both body text and math, otherwise use something else as requested by instructor
\usepackage{newpxtext,newpxmath}

\DeclareLanguageMapping{american}{american-mla}
\newdateformat{dateMLA}{\THEDAY\ \monthname[\THEMONTH] \THEYEAR} % MLA has a specific date standard

\setstretch{2.25} % Experimentally derived spacing for word doc style. Not perfect, but at least lines per page are consistant
\setlength{\parskip}{8pt} % Typically, Word defaults to 8pt line skips, which here is the 'de facto' (much like line-spacing)

\fancyhf{}
\rhead{\lastname~\thepage}

\renewcommand{\headrulewidth}{0pt} % No headrules for the headers and footers, unless you (and your instructor) are into that...
\renewcommand{\footrulewidth}{0pt}

% Ensure header 0.5in from top edge: 12pt font is 1/6 inch
\setlength\headsep{0.333in}
% Set the header height (complains otherwise)
\setlength\headheight{14.49998pt}

\pagestyle{fancy} % Every page uses the same style.

% Custom format of bibliography for MLA
\newcommand{\printformatedbibliography}{\newpage\begin{center}Works Cited\end{center}\printbibliography[heading=none]}

% Per MLA requirements, endnotes should appear on a extra page before the bibliography with a header.
\newcommand{\printnotes}{\newpage\printendnotes}

% Useful if you want to introduce an author by their full name initially.
\newrobustcmd*{\citefullname}{\AtNextCite{\DeclareNameAlias{labelname}{given-family}}\citeauthor} 

\newenvironment{paper}{
\firstname~\lastname \\ 
\instructorfull\\ 
\className
% \dateMLA{\today}
% \begin{center}\titleName\end{center}
\setlength\parindent{0.5in}
}{
%\printnotes % If your not using endnotes, just comment this out...
\printformatedbibliography % Most likely, your should keep this
}

\setlength\parindent{0.5in}
\raggedright


% The "post-include" section. Simply include your bibliography database file and begin the document. Below is the example code
%
% \bibliography{sources.bib} %<-- BTY, you have to do this post-include.
% \begin{document}
% \begin{paper}
% Here be a footnote \footnote{EEEEE! Sup}. Here be a endnote\endnote{EEEEEEEEEEEEEEEEEE! We dont use those often}
% \end{paper}
% \end{document}
% 
% 
\usepackage[american]{babel} 
\usepackage{csquotes} 
\usepackage[hidelinks]{hyperref} % Links otherwise get in the way, so we hide them. Do want you want...
\usepackage[style=mla-new, backend=biber]{biblatex} % Your weird for not using this stuff...(if not why are you really using latex)
\usepackage{datetime}
\usepackage[letterpaper, margin=1.0in]{geometry} %Per mla suggestion, but some instructors deviate...
\usepackage{fancyhdr}
\usepackage{setspace}
\usepackage{enotez} % The modern version of the "endnotes" package. 
\setenotez{backref, counter-format=alph, list-heading = \begin{center}#1\end{center}} % Here we enable backlinks and switch the counter to alphabetical since usually there are less end-notes and to differentiate between them and footnotes. Also a neat trick, just change the list-heading field to be unformated and centered per MLA standard.

% We like Zapf's Palatino font for both body text and math, otherwise use something else as requested by instructor
\usepackage{newpxtext,newpxmath}

\DeclareLanguageMapping{american}{american-mla}
\newdateformat{dateMLA}{\THEDAY\ \monthname[\THEMONTH] \THEYEAR} % MLA has a specific date standard

\setstretch{2.25} % Experimentally derived spacing for word doc style. Not perfect, but at least lines per page are consistant
\setlength{\parskip}{8pt} % Typically, Word defaults to 8pt line skips, which here is the 'de facto' (much like line-spacing)

\fancyhf{}
\rhead{\lastname~\thepage}

\renewcommand{\headrulewidth}{0pt} % No headrules for the headers and footers, unless you (and your instructor) are into that...
\renewcommand{\footrulewidth}{0pt}

% Ensure header 0.5in from top edge: 12pt font is 1/6 inch
\setlength\headsep{0.333in}
% Set the header height (complains otherwise)
\setlength\headheight{14.49998pt}

\pagestyle{fancy} % Every page uses the same style.

% Custom format of bibliography for MLA
\newcommand{\printformatedbibliography}{\newpage\begin{center}Works Cited\end{center}\printbibliography[heading=none]}

% Per MLA requirements, endnotes should appear on a extra page before the bibliography with a header.
\newcommand{\printnotes}{\newpage\printendnotes}

% Useful if you want to introduce an author by their full name initially.
\newrobustcmd*{\citefullname}{\AtNextCite{\DeclareNameAlias{labelname}{given-family}}\citeauthor} 

\newenvironment{paper}{
\firstname~\lastname \\ 
\instructorfull\\ 
\className
% \dateMLA{\today}
% \begin{center}\titleName\end{center}
\setlength\parindent{0.5in}
}{
%\printnotes % If your not using endnotes, just comment this out...
\printformatedbibliography % Most likely, your should keep this
}

\setlength\parindent{0.5in}
\raggedright


% The "post-include" section. Simply include your bibliography database file and begin the document. Below is the example code
%
% \bibliography{sources.bib} %<-- BTY, you have to do this post-include.
% \begin{document}
% \begin{paper}
% Here be a footnote \footnote{EEEEE! Sup}. Here be a endnote\endnote{EEEEEEEEEEEEEEEEEE! We dont use those often}
% \end{paper}
% \end{document}
% 
% 
\usepackage[american]{babel} 
\usepackage{csquotes} 
\usepackage[hidelinks]{hyperref} % Links otherwise get in the way, so we hide them. Do want you want...
\usepackage[style=mla-new, backend=biber]{biblatex} % Your weird for not using this stuff...(if not why are you really using latex)
\usepackage{datetime}
\usepackage[letterpaper, margin=1.0in]{geometry} %Per mla suggestion, but some instructors deviate...
\usepackage{fancyhdr}
\usepackage{setspace}
\usepackage{enotez} % The modern version of the "endnotes" package. 
\setenotez{backref, counter-format=alph, list-heading = \begin{center}#1\end{center}} % Here we enable backlinks and switch the counter to alphabetical since usually there are less end-notes and to differentiate between them and footnotes. Also a neat trick, just change the list-heading field to be unformated and centered per MLA standard.

% We like Zapf's Palatino font for both body text and math, otherwise use something else as requested by instructor
\usepackage{newpxtext,newpxmath}

\DeclareLanguageMapping{american}{american-mla}
\newdateformat{dateMLA}{\THEDAY\ \monthname[\THEMONTH] \THEYEAR} % MLA has a specific date standard

\setstretch{2.25} % Experimentally derived spacing for word doc style. Not perfect, but at least lines per page are consistant
\setlength{\parskip}{8pt} % Typically, Word defaults to 8pt line skips, which here is the 'de facto' (much like line-spacing)

\fancyhf{}
\rhead{\lastname~\thepage}

\renewcommand{\headrulewidth}{0pt} % No headrules for the headers and footers, unless you (and your instructor) are into that...
\renewcommand{\footrulewidth}{0pt}

% Ensure header 0.5in from top edge: 12pt font is 1/6 inch
\setlength\headsep{0.333in}
% Set the header height (complains otherwise)
\setlength\headheight{14.49998pt}

\pagestyle{fancy} % Every page uses the same style.

% Custom format of bibliography for MLA
\newcommand{\printformatedbibliography}{\newpage\begin{center}Works Cited\end{center}\printbibliography[heading=none]}

% Per MLA requirements, endnotes should appear on a extra page before the bibliography with a header.
\newcommand{\printnotes}{\newpage\printendnotes}

% Useful if you want to introduce an author by their full name initially.
\newrobustcmd*{\citefullname}{\AtNextCite{\DeclareNameAlias{labelname}{given-family}}\citeauthor} 

\newenvironment{paper}{
\firstname~\lastname \\ 
\instructorfull\\ 
\className
% \dateMLA{\today}
% \begin{center}\titleName\end{center}
\setlength\parindent{0.5in}
}{
%\printnotes % If your not using endnotes, just comment this out...
\printformatedbibliography % Most likely, your should keep this
}

\setlength\parindent{0.5in}
\raggedright


% The "post-include" section. Simply include your bibliography database file and begin the document. Below is the example code
%
% \bibliography{sources.bib} %<-- BTY, you have to do this post-include.
% \begin{document}
% \begin{paper}
% Here be a footnote \footnote{EEEEE! Sup}. Here be a endnote\endnote{EEEEEEEEEEEEEEEEEE! We dont use those often}
% \end{paper}
% \end{document}
% 